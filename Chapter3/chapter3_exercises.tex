\documentclass{article}
\usepackage{amsthm}
\usepackage{mathtools}
\usepackage{fullpage}
\usepackage{amsfonts}

\newcommand{\lined}{\noindent\rule{\textwidth}{1pt}}
\newtheorem*{problem}{Problem}
\newtheorem*{disproof}{\emph{Disproof}}
\newtheorem*{remark}{Remark}





\title{\huge Analysis 1: Chapter 3 \\ \LARGE {Set Theory Exercises}}
\date{}

\begin{document}
	\maketitle
	
	\begin{remark}
		In the text Tao informally gives a way to resolve Russell's paradox. He describes that objects are arranged (at least in set theory)
		in an hierarchy. At the first rung you have (impure set theory) what are called primitive objects which are not sets. At the next 
		rung you have sets which only contain primitive objects. The rung after that has sets which only contain primitive objects or sets containing only primitive objects and so on. From this you see that if you are on a rung of the hierarchy not at the primitive object level then all sets at that level can only contain objects below it on the hierarchy. So things like a set containing itself
		cant be true because that set is on the same rung as itself. That is the intuition for the Axiom of Regularity and the problem below.
	\end{remark}
	
	\begin{problem}[Exercise 3.2.2]
		Let $A$ be a set. Show that $A \notin A$. Moreover show that for any two sets $A$ and $B$ that $B \notin A$ or $A \notin B$ (or both).
	\end{problem}

	\lined
	\begin{proof}
		We shall prove the first part that is no set contains itself. If $A$ is the empty set then we are done. Suppose now then that $A$
		is non-empty and for contradiction that $A \in A$. Using the singleton set axiom we have the singleton set $\{A\}$ exists. However if $A \in A$ then $A \in \{A\}$ but $A$ is a set that is not disjoint from $\{A\}$. Thus $\{A\}$ is a non-empty set that
		contains just the set $A$ which is not disjoint from itself. This contradicts the Axiom of Regularity thus for any set $A$ we must have $A \notin A$.
		\linebreak
		
		\noindent We now prove the second part. Let $A$ and $B$ be two sets. If $A = B$ then from the first part we have that
		 $B \notin A$ and $A \notin B$ and we see the result is true in this case. Moreover if $A$ or $B$ are the empty set then the result follows. Thus we suppose that $A$ and $B$ are two non-empty and non-equal sets. Suppose for contradiction that 
		 $A \in B$ and $B \in A$. Using the pair set axiom we can construct the pair set $\{A,B\}$ which is a non-empty set that only contains $A$ and$B$. That is for all objects $z$ we have $z \in \{A,B\}$ if and only if $z = A$ or $z = B$. In either case we have that $z \cap \{A,B\} \neq \emptyset$ thus $\{A,B\}$ is a non-empty that only contains sets that are not disjoint from itself. This contradicts the Axiom of Regularity thus we must have $A \notin B$ or $B \notin A$. The result follows as desired.
	\end{proof}
	\lined
	
	\newpage
	
	\begin{problem}[Exercise 3.4.6]
			Let $X$ be a set. Show that the collection of all subsets of $X$ is a set.
	\end{problem}
	
	\lined
		\begin{proof}
		Appealing to the Axiom of Infinity and the Pair Set Axiom we have that $\{0,1\}$ is a set. 
		By the Power Set Axiom there exists a set containing all functions from $X$ to $\{0,1\}$ which
		we shall denote as $\{0,1\}^X$. Define the property $P(f,S)$ pertaining to each $f \in \{0,1\}^X$ and
		any object $S$ to be the statement $S$ is a set and $S = f^{-1}(\{1\})$.  There is at most one set $S$ 
		such that $P(f,S)$ is true for each $f \in \{0,1\}^X$. Thus by the Axiom of Replacement
		$Q \coloneqq \{S:$ $S = f^{-1}(\{1\})$ for some $f \in \{0,1\}^X\}$ exists and is a set. All that remains to show
		is that every element of $Q$ is a subset of $X$ and every subset of $X$ is contained in $Q$. Let $Z$ be an arbitrary
		element of $Q$, then there exists some function $f \in \{0,1\}^X$ such that $Z = f^{-1}(\{1\})$ which is indeed a subset of
		$X$. Conversely let $Z$ be a subset of $X$. We shall show that there exists some function 
		$f \in \{0,1\}^X$ such that $Z = f^{-1}(\{1\})$. For each $x \in X$ define $f$ to be the mapping such that 
		$f(x) \coloneqq 1$ if $x \in Z$ and $f(x) \coloneqq 0$ if $x \notin Z$. This is clearly a function from $X$ to $\{0,1\}$ which has
		$Z = f^{-1}(\{1\})$ thus $Z \in Q$ and the result follows.
		
	\end{proof}
	\lined
	
	\newpage
	
	\begin{problem}[Exercise 3.4.7]
		Let $X$ and $Y$ be sets. A function $f:X' \rightarrow Y'$ from a subset $X'$ of $X$ to a subset $Y'$ of $Y$ is
		said to be a partial function from $X$ to $Y$. If $X$ and $Y$ are sets show that the collection of all partial functions
		from $X$ to $Y$ is a set.
	\end{problem}

	\begin{remark}
		The idea for the proof we shall give is analogous to a sort of double looping construct procedure in a
		computer programming language. In particular let $2^X$ and $2^Y$ be the collections of all subsets of $X$ and
		$Y$ respectively which are sets by the previous exercise. Informally, or each element of $2^X$ we loop through each element of
		$2^Y$ creating a set containing all sets of function spaces where a function space is just the set of all functions from some
		particular element of $2^X$ to some particular element of $2^Y$. Then we use the Axiom of Union to unbox all these sets of function spaces to get the set of all function spaces. Unboxing these sets in the same manner we get the set of all partial functions from $X$ to $Y$	\end{remark}

	\lined
		\begin{proof}
			Since $X$ and $Y$ are sets by the previous exercise asserts that the collection of all subsets of $X$ is a set and also that
			the collection of all subsets of $Y$ is set. Let $2^X$ and $2^Y$ denote these sets respectively. For each $X' \in 2^X$ and any object S define the property $P(X',S)$ to be the statement $S$ is a set such that for all objects $z$, we have 
			
			\[z \in S \iff z = Y'^{X'}\ \text{for some}\  Y' \in 2^Y.\]
			
			\noindent There is at most one set $S$ such that $P(X',S)$ is true for each $X' \in 2^X$. Thus by the Axiom of Replacement there exists a set $Q \coloneqq \{S:P(X',S)\ \text{is true for some}\ X' \in 2^X\}$. We observe that $Q$ is a family of sets thus
			by the axiom of the union the set $\bigcup Q$ exists. This set like $Q$ is also a family of sets. Thus applying the Axiom of Union once again we have that the set $\bigcup(\bigcup Q)$ exists. We claim that every element of $\bigcup(\bigcup Q)$ is a partial function from $X$ to $Y$ and that every partial function from $X$ to $Y$ is contained in $\bigcup(\bigcup Q)$.
			Let $f'$ be an arbitrary element of $\bigcup(\bigcup Q)$ then $f' \in T$ for some $T \in \bigcup Q$. But then any such $T$ will be contained in $H$ for	some set $H \in Q$. However if $H \in Q$ then for some $X' \in 2^X$	 we have for every object $K$ that $K \in H$ if and only if $K = Y'^{X'}\ \text{for some}\  Y' \in 2^Y$. Thus since $T \in H$ we have $T = Y'^{X'}$ for some 
			 $Y' \in 2^Y$. So that if $f' \in T$ then $f'$ is a partial function from $X$ to $Y$. We therefore see that every element of $\bigcup(\bigcup Q)$ is a partial function from $X$ to $Y$. Conversely let $f':X' \rightarrow Y'$ be a partial function from $X$ to $Y$ from a subset $X'$ of $X$ to a subset $Y'$ of $Y$. We shall show that $f' \in \bigcup(\bigcup Q)$. First note that by the Power Set Axiom the set $Y'^{X'}$ exists. Now for each $Y'' \in 2^Y$ let $P'(Y'',S')$ be the statement
			 $S'$ is a set such that $S' = Y''^{X'}$. For each $Y'' \in 2^Y$ there is at most one such set $S'$ thus by the Axiom of Replacement the set $H \coloneqq \{S':S' = Y''^{X'}\ \text{for some}\ Y'' \in 2^Y \}$ exists. Now observe that  $Y'^{X'} \in H$ and $H \in Q$ which implies that $Y'^{X'} \in \bigcup Q$ but $f' \in Y'^{X'}$ thus $f' \in \bigcup(\bigcup Q)$ and the result follows.
			 
		\end{proof}
	\lined
	
	\newpage
	
	\begin{remark}
		Exercise 3.5.1 in the third edition of the text consists of three parts even though all are listed as
		a single exercise.
	\end{remark}

	\begin{problem}[Exercise 3.5.1 (a)]
		Let $x$ and $y$ be any two objects (possibly the same). We define $(x,y) \coloneqq  \{\{x\},\{x,y\}\}$. Which can be shown to be a set that exists using Axiom 3.4 and the Axiom of Union. Show that $(x,y)$ obeys the usual property we would expect of an ordered pair (i.e two ordered pairs are equal if and only if all their components are equal).
	\end{problem}

	\lined
	\begin{proof}
		We shall show that for any objects $a,a',b\ \text{and}\  b'$ that $(a,b) = (a',b') \iff (a = a'\ \text{and}\ b = b')$.
		$(\implies)$ Suppose that $(a,b) = (a',b')$ then by definition we have that $\{\{a\},\{a,b\}\} = \{\{a'\},\{a',b'\}\}$. We will first show that $a = a'$. Since $\{a\} \in \{\{a\},\{a,b\}\}$ we have $\{a\} = \{a'\}$ or $\{a\} = \{a',b'\}$. If $\{a\} = \{a'\}$ then we are done. On the other hand if $\{a\} \neq \{a'\}$ then we must have $\{a\} = \{a',b'\}$ which implies that $\{a\} = \{a'\}$. Therefore in either case $\{a\} = \{a'\}$. Now we show that $\{b\} = \{b'\}$. From our hypothesis we have that $\{a,b\} \in \{\{a'\},\{a',b'\}\}$ and thus that
		$\{a,b\} = \{a'\}$ or $\{a,b\} = \{a',b'\}$. If $\{a,b\} = \{a'\}$ then $a = b = a'$. However $\{a',b'\} \in \{\{a\},\{a,b\}\}$ by our hypothesis so that $\{a',b'\} = \{a\}$ or $\{a',b'\} = \{a,b\}$ and in either case we have that $b' = b$. If on the other hand we have
		$\{a,b\} \neq \{a'\}$ then we must have $\{a,b\} = \{a',b'\}$ which implies that $b' = a$ or $b = b'$. If $b = b'$ we are done so we  consider the case when $b' = a$. Since $\{a,b\} \in \{\{a'\},\{a',b'\}\}$ we have that 
		$\{a,b\} = \{a'\}$ or $\{a,b\} = \{a',b'\}$ and in either case we have $b = b'$ since $a = a' = b'$.
		\linebreak
		
		\noindent $(\impliedby)$ The converse is trivial
	\end{proof}
	\lined
	
	\newpage
	
	\begin{remark}
		We will appeal to the previous part of this exercise (i.e we will use the result and definition of ordered pair in part a). The proof
		we will provide is similar to that of Exercise 3.4.7. 
	\end{remark}

	\begin{problem}[Exercise 3.5.1 (b)]
		Let $X$ and $Y$ be two sets. Show that the Cartesian Product of $X$ and $Y$ is a set (i.e that a set containing
		all ordered pairs whose first component is from $X$ and second is from $Y$ exists).
	\end{problem}

	\lined
	\begin{proof}
		For each $x \in X$ and any object $S$ define $P(x,S)$ to be the statement $S$ is a set such that for all objects $z$ we have
		$z \in S \iff z = \{\{x\},\{x,y\}\}$ for some $y \in Y$. For each $x \in X$ there is at most one set $S$ such that $P(x,S)$ is true. Thus by the Axiom of Replacement there is a set $\{S:P(x,S)\ \text{is true for some}\ x \in X\}$. By the Axiom of 
		Union we have that $Q \coloneqq \bigcup (\{S:P(x,S)\ \text{is true for some}\ x \in X\})$ is a set. We claim that every element of $Q$ is an ordered pair $(x,y)$ where $x \in X$ and $y \in Y$ and that every such ordered pair is in $Q$. First suppose that $z \in Q$ then $z \in T$ for some $T \in \{S:P(x,S)\ \text{is true for some}\ x \in X\}$. But then for this $T$ there exists some $x \in X$ such that for every object $q$ we have $q \in T$ if and only if $q = \{\{x\},\{x,y\}\}$ for some $y \in Y$. Thus we have $z = \{\{x\},\{x,y\}\}$ for some $x \in X$ and $y \in Y$ but by definition of an ordered pair for this $x$ and $y$ we have $(x,y) = \{\{x\},\{x,y\}\} = z$ thus every element of $Q$ is an ordered pair $(x,y)$ where $x \in X$ and $y \in Y$. Now suppose that $(x',y')$ is an ordered pair with $x' \in X$ and $y' \in Y$. For each $y \in Y$ and any object $q$ define $P'(y,q)$ to be the statement $q = \{\{x'\},\{x',y\}\}$. For each $y \in Y$ there is at most one object $q$ such that $P'(y,q)$ is true. Thus be the Axiom of Replacement there is a set $T \coloneqq \{q:P'(y,q)\ \text{is true for some}\ y \in Y\}$. Observe then that $P(x',T)$ is true thus $T \in \{S:P(x,S)\ \text{is true for some}\ x \in X\}$. Moreover $(x',y') = \{\{x'\},\{x',y'\}\} \in T$ thus $(x',y') \in Q$. The result follows as desired.
		 
	\end{proof}
	\lined
	
	\newpage
	
	\begin{remark}
		See the texts definition for an ordered pair and observe we do not assert that an ordered pair is a set. However if we take the definition of an ordered set as it is in the exercise below it satisfies the property of what we would expect an ordered pair to have (i.e that two ordered pairs are equal if and only if their first and second components are the same).
	\end{remark}
	
	\begin{problem}[Exercise 3.5.1 (c)]
		Show that for any two objects $x$ and $y$ (possibly the same) that $(x,y) \coloneqq \{x,\{x,y\}\}$ can also be used as
		an alternative definition for an ordered pair.
	\end{problem}

	\lined
	\begin{proof}
		We shall show that for any objects $a,a',b$ and $b'$ that $\{a,\{a,b\}\} = \{a',\{a',b'\}\}$ if and only if $a = a'$ and
		$b = b'$. ($\implies$) Suppose that  $\{a,\{a,b\}\} = \{a',\{a',b'\}\}$. We shall first show that $a = a'$. Suppose for contradiction
		that $a \neq a'$ then we must have that $a = \{a',b'\}$ and $a' = \{a,b\}$. However in this situation we will then have that
		$a \in a'$ and $a' \in a$  contradicting Exercise 3.2.2. Thus our initial assumption was false and we must have $a = a'$.  Now we shall show that $b = b'$. By assumption we have $\{a,b\} \in \{a',\{a',b'\}\}$ so that $a' = \{a,b\}$ or $\{a',b'\} = \{a,b\}$. We have that $\{a,b\} \neq a'$ since otherwise we shall have $a = \{a,b\}$ as a direct consequence of the fact $a = a'$ which we have shown already. However then we have that $a \in a$ contradicting the fact that for any set $A$ we must have $A \notin A$ (see Exercise 3.2.2). Therefore $\{a,b\} \neq a'$ but then $\{a,b\} = \{a',b'\}$. We see then that $b = a'$ or $b = b'$. If $b = b'$ we have nothing to show and if that is not the case then $b = a'$ but $b' = a$ and $a = a'$ thus $b = b'$. In either case we have $b = b'$ as desired.
		\linebreak
		\noindent ($\impliedby$) The converse is trivial. 
	\end{proof}
	\lined
	
	\newpage
	
	\begin{remark}
		Before we provide a solution for Exercise 3.5.2 we will discuss shortly empty functions. A empty function to an arbitrary set $X$ is defined to be a function $f:\emptyset \rightarrow X$ whose domain is the empty set and range is $X$. In the text Tao states that given any arbitrary set $X$ there is exactly one empty function to $X$. We will not concern ourselves with the general case but rather with the specific case when $X$ is the empty set. Note that in this case we immediately have that $f:\emptyset \rightarrow \emptyset$ is a function no matter how we define $f$. Since for each $x \in \emptyset$ we have exactly one $y \in \emptyset$ assigned to $x$ (this is vacuously true). So we do not need to specify what $f$ does. Furthermore any other function from $g:\emptyset \rightarrow \emptyset$ will be equal to $f:\emptyset \rightarrow \emptyset$ since their domains and ranges are the empty set and for each $x \in \emptyset$ we have $f(x) = g(x)$ vacuously. Thus $f:\emptyset \rightarrow \emptyset$ is the unique empty function to the empty set. Note that this $f$ is also surjective (vacuously again) and as you'll see when a definition for an ordered n-tuple is given in Exercise 3.5.2 that this $f$ is the only 0-tuple with respect to that given definition. Thus we are justified in calling $f$ the 0-tuple or the empty tuple which we will denote as $( )$.
	\end{remark}


	\begin{problem}[Exercise 3.5.2 (a)]
		Let $n$ be a natural number. Suppose we define an ordered n-tuple to be a surjective function $x:\{i \in \mathbb{N}:1 \leq i \leq n\} \rightarrow X$ where $X$ is an arbitrary set (different n-tuples are allowed to have different ranges). We define the $i$-th component of $x$ denoted by $x_i$ to be $x(i)$, and we also denote $x$ as $(x_i)_{1 \leq i \leq n}$. Show that for two ordered $n$-tuples $x$ and $y$ that $(x_i)_{1 \leq i \leq n} = (y_i)_{1 \leq i \leq n}$ if and only if $x(i) = y(i)$ for $1 \leq i \leq n$.
	\end{problem}
	
	\lined
	\begin{proof}
		We first consider the special case when $n = 0$. A $0$-tuple is a surjective function $x:\{i \in \mathbb{N}:1 \leq i \leq 0\} 
		\rightarrow X$. However $\{i \in \mathbb{N}:1 \leq i \leq 0\} = \emptyset$. Thus $x$ must be an empty function to some set $X$.
		$x$ cannot be surjective unless $X$ is the empty set thus any $0$-tuple must an empty function to the empty set. However from our previous discussion we know that there is exactly one function from the empty set to the empty set so we have exactly
		one $0$-tuple. So the result is true when $n = 0$. We now suppose that $n \geq 1$. Let $x$ and $y$ be two ordered $n$-tuples. ($\implies$) Suppose that $(x_i)_{1 \leq i \leq n}$ = $(y_i)_{1 \leq i \leq n}$. Then since $x$ and $y$ are equal functions they have the same domain $\{i \in \mathbb{N}:1\leq i \leq n\}$ and range $X$ such that for each $i \in \{i \in \mathbb{N}:1 \leq i \leq n\}$ we have $x(i) = y(i)$. Let $i$ be a natural number between $1$ and $n$, then $i \in \{i \in \mathbb{N}:1 \leq i \leq n\}$  thus $x(i) = y(i)$ for all $1 \leq i \leq n$.
		\linebreak
		
		\noindent ($\impliedby$) Suppose that $x$ are $y$ are two ordered $n$-tuples such that $x(i) = y(i)$ for all $1 \leq i \leq n$. Since they both have the same domain $\{i \in \mathbb{N}:1 \leq i \leq n\}$, we need to show that they have the same range and that they agree for all elements in their shared domain. Let $X$ and $X'$ denote the range of $x$ and $y$ respectively. Suppose that $z$ is an arbitrary element of $X$, then since $x$ is surjective there is some $i$ in $\{i \in \mathbb{N}:1 \leq i \leq n\}$
		such that $z = x(i)$ however $i$ is an integer between $1$ and $n$ thus by assumption we have $x(i) = y(i)$ and thus $z \in X'$.
		Using the same argument but with the roles of $X$ and $X'$ swapped and the roles of $x$ and $y$ swapped we have that if $z$ is an element of $X'$ then $z$ is also an element of $X$. Therefore $X$ and $X'$ are equal thus $x$ and $y$ have the same range. Let $i \in \{i \in \mathbb{N}:1 \leq i \leq n\}$ then $i$ is an integer between $1$ and $n$ thus by assumption $x(i) = y(i)$. The result follows as desired.
	\end{proof}
	\lined
	
	\newpage
	
	\begin{problem}[Exercise 3.5.2 (b)]
		Using the definition of an $n$-tuple in part (a) show that if $(X_i)_{1 \leq i \leq n}$ is an ordered $n$-tuple then 
		$\prod_{1 \leq i \leq n} X_i$ is a set (satisfying our definition in the text).
	\end{problem}
	
	\lined
	\begin{proof}
		Suppose that $(X_i)_{1 \leq i \leq n}$ is an ordered $n$-tuple. Then we have that $(X_i)_{1 \leq i \leq n}$ is a surjective function
		from $\{i \in \mathbb{N}:1\leq i \leq n\}$ to some set of sets $T$. Appealing to the Axiom of Union we have that the set 
		$\bigcup T$ exists. Then using the result we proved in Exercise 3.4.7 the collection of all partial functions from $\{i \in \mathbb{N}:1\leq i \leq n\}$ to $\bigcup T$ is a set which we shall denote as $P$. For each $x$ in $P$ define $P'(x)$ to be the statement $x$ is surjective with domain $\{i \in \mathbb{N}:1\leq i \leq n\}$ and range $T' \subseteq \bigcup T$ such that for each $i$ in $\{i \in \mathbb{N}:1\leq i \leq n\}$ we have $x(i) \in X_i$. Then the Axiom of Specification gives us the set 
		$\prod_{1 \leq i \leq n} X_i \coloneqq \{x:P'(x)\ \text{is true}\}$. We claim that every every ordered $n$-tuple $(x_i)_{1 \leq i \leq n}$
		where $x_i \in X_i$ for $1 \leq i \leq n$ is in $\prod_{1 \leq i \leq n} X_i$ and that every element of $\prod_{1 \leq i \leq n} X_i$ is such an $n$-tuple. Let $z$ be an arbitrary element of $\prod_{1 \leq i \leq n} X_i$ then $z \in P$ and $P'(z)$ is true. That implies that $z$ is a surjective function with domain $\{i \in \mathbb{N}:1\leq i \leq n\}$ and range $T' \subseteq \bigcup T$ such that for each $i$ in $\{i \in \mathbb{N}:1\leq i \leq n\}$ we have $z(i) \in X_i$. But then $z$ is precisely an ordered $n$-tuple such that 
		$x_i \in X_i$ for $1 \leq i \leq n$. Now suppose that $(x_i)_{1 \leq i \leq n}$ is an ordered $n$-tuple such that $x_i \in X_i$ for $1 \leq i \leq n$. We note that $(x_i)_{1 \leq i \leq n}$ has domain $\{i \in \mathbb{N}:1\leq i \leq n\}$ and set $T'$ as its range. We claim that $T' \subseteq \bigcup T$. Let $t \in T'$ then since $x$ is surjective we have some $i \in \{i \in \mathbb{N}:1\leq i \leq n\}$
		such that $t = x_i$ however $x_i \in X_i$ thus $t \in \bigcup T$ and therefore $T' \subseteq \bigcup T$. We have thus shown $(x_i)_{1 \leq i \leq n} \in P$ and we also have $P'(x)$ is true thus $(x_i)_{1 \leq i \leq n} \in \prod_{1 \leq i \leq n} X_i$. The result follows as required.
	\end{proof}
	\lined


	\newpage
	
	\begin{remark}
		In the following two exercises we have to assume that the objects of our ordered $n$-tuples and ordered pairs
		obey the axioms of transitivity, reflexivity and symmetry.  
	\end{remark}
	
	\begin{problem}[Exercise 3.5.3 (a)]
		Show that the definition of an ordered pair obeys the reflexivity, symmetry and transitivity axioms.
	\end{problem}
	
	\lined
	\begin{proof}
		(Reflexivity) Let $(a,b)$, $(a',b')$ and $(x,y)$ be ordered pairs. Then applying the axioms of equality directly to the objects $a$ and $b$ we have $a = a$ and $b = b$ thus $(a,b) = (a,b)$. (Symmetry) Suppose that $(a,b) = (a',b')$ then $a = a'$ and $b = b'$ but applying the axioms of equality directly to the objects $a,a',b$ and $b'$ we have $a' = a$ and $b' = b$ whence 
		$(a',b') = (a,b)$. (Transitivity) Suppose $(a,b) = (a',b')$ and $(a',b') = (x,y)$. Then we have $a = a'$ and $a' = x$. Applying the axioms of equality directly to these objects we have $a = x$. A similar argument shows that $b = y$ thus $(a,b) = (x,y)$. The result follows as desired.
		
	\end{proof}
	\lined
	
	\newpage
	
	\begin{problem}[Exercise 3.5.3 (b)]
		Show that the definition of ordered $n$-tuples obeys the reflexivity, symmetry and transitivity axioms.
	\end{problem}
	
	\lined
	\begin{proof}
		Very similar to the proof of part (a)
	\end{proof}
	\lined
	
	\newpage
	
	\begin{problem}[Exercise 3.5.4 (a)]
		Let $A,B$ and $C$ be sets. Show that $A \times (B \cup C) = (A \times B) \cup (A \times C)$.
	\end{problem}
	
	\lined
	\begin{proof}
		Let $z$ be an arbitrary element of $A \times (B \cup C)$ then $z$ is an ordered pair $(x,y)$ with $x \in A$ and 
		$y \in B \cup C$. We have that $y \in B$ or $y \in C$. In either situation $y \in B$ or $y \notin B$ we have that $z \in 
		(A \times B) \cup (A \times C)$. Therefore $A \times (B \cup C) \subseteq (A \times B) \cup (A \times C)$. Now suppose that $z$ is an arbitrary element of  $(A \times B) \cup (A \times C)$ then $z \in A \times B$ or $z \in A \times C$. If $z \in A \times B$ or $z \notin A \times B$  then $z \in A \times (B \cup C)$. Thus $(A \times B) \cup (A \times C) \subseteq A \times (B \cup C)$. The result follows as desired.
	\end{proof}
	\lined
	
	\newpage
	
	\begin{problem}[Exercise 3.5.4 (b)]
		Let $A,B$ and $C$ be sets. Show that $A \times (B \cap C) = (A \times B) \cap (A \times C)$.
	\end{problem}
	
	\lined
	\begin{proof}
		Let $z$ be an arbitrary element of $A \times (B \cap C)$ then $z$ is an ordered pair $(x,y)$ where $x \in A$ and 
		$y \in B \cap C$. Which implies that $z \in (A \times B) \cap (A \times C)$. Now suppose that $z$ is an element of
		$(A \times B) \cap (A \times C)$ then we have $z$ is an ordered pair $(x,y)$ with $x$ in $A$ and $y \in B \cap C$ which implies that $z \in A \times (B \cap C)$ and the result follows as desired.
	\end{proof}
	\lined
	
	\newpage 
	
	\begin{problem}[Exercise 3.5.4 (c)]
		Let $A,B$ and $C$ be sets. Show that $A \times (B \backslash C) = (A \times B) \backslash (A \times C)$.
	\end{problem}

	\lined
	\begin{proof}
		Let $z$ be an arbitrary element of $A \times (B \backslash C)$ then $z$ is an ordered pair $(x,y)$ with $x \in A$ but
		$y \in B$ and $y \notin C$. Thus $z \in A \times B$ but $z$ is not in $A \times C$ so $z \in (A \times B) \backslash (A \times C)$.
		Now suppose that $z$ is an arbitrary element of $(A \times B) \backslash (A \times C)$. Then $z$ is an ordered pair $(x,y)$ in 
		$A \times B$ but not in $A \times C$. However since $z \notin A \times C$ and $x \in A$ we must have that $y \notin C$. 
		Therefore $y \in B \backslash C$ so $z \in A \times (B \backslash C)$. The result follows as desired.
	\end{proof}
	\lined
	
	\newpage
	
	\begin{problem}[Exercise 3.5.5 (a)]
		Let $A,B,C,$ and $D$ be sets. Show that $(A \times B) \cap (C \times D) = (A \cap C) \times (B \cap D)$.
	\end{problem}
	
	
	\lined
	\begin{proof}
		Let $z$ be an element of $(A \times B) \cap (C \times D)$ then $z$ is an ordered pair $(x,y)$ such that 
		$x \in A$ and $x \in C$ and $y \in B$ and $y \in D$. However then $z$ is in $(A \cap C) \times (B \cap D)$.  Now
		suppose that $z$ is an element of $(A \cap C) \times (B \cap D)$ then $z$ is an ordered pair $(x,y)$ such that $x \in A \cap C$
		and $y \in B \cap D$ which implies that $z \in A \times B$ and $z \in C \times D$. The result follows as desired.
	\end{proof}
	\lined
	
	\newpage
	
	\begin{problem}[Exercise 3.5.5 (b)]
		Let $A,B,C,$ and $D$ be sets. Prove or disprove that $(A \times B) \cup (C \times D) = (A \cup C) \times (B \cup D)$.
	\end{problem}
	
	\lined
	\begin{disproof}
		In general it is true that $(A \times B) \cup (C \times D) \subseteq (A \cup C) \times (B \cup D)$. On the other hand we have that
		$(A \cup C) \times (B \cup D) \subseteq (A \times B) \cup (C \times D)$ is not true in general.  We can obtain a counter example by setting $A \coloneqq \{1,2\}$, $B \coloneqq \{3,4\}$, $C \coloneqq \{5,6\}$, and $D \coloneqq \{7,8\}$. We observe that $(1,7) \in (A \cup C) \times (B \cup D)$ but  $(1,7) \notin (A \times B) \cup (C \times D)$. Therefore $(A \times B) \cup (C \times D) = (A \cup C) \times (B \cup D)$ is not true in general.
	\end{disproof}
	\lined
	
	\newpage
	
	\begin{problem}[Exercise 3.5.5 (c)]
		Let $A,B,C,$ and $D$ be sets. Prove or disprove that $(A \times B) \backslash (C \times D) = (A \backslash C) \times (B \backslash D)$.
	\end{problem}

	\lined
	\begin{disproof}
		In general we do have $(A \backslash C) \times (B \backslash D) \subseteq (A \times B) \backslash (C \times D)$ but in general we do not have $(A \times B) \backslash (C \times D) \subseteq (A \backslash C) \times (B \backslash D)$. We can obtain a counter example by setting $A \coloneqq \{1,2\}$, $B \coloneqq \{5,7\}$, $C \coloneqq \{1,3\}$, and $D \coloneqq \{6,9\}$. But then we have $(1,5) \in (A \times B) \backslash (C \times D)$ but $(1,5) \notin (A \backslash C) \times (B \backslash D)$. Thus in general
		$(A \times B) \backslash (C \times D) = (A \backslash C) \times (B \backslash D)$ is not true. 
	\end{disproof}
	\lined
	
	\newpage
	
	\begin{problem}[Exercise 3.5.6]
		Let $A,B,C,D$ be non-empty sets. Show that $A \times B \subseteq C \times D$ if and only if $A \subseteq C$ and $B \subseteq
		D$.
	\end{problem}

	\lined
	\begin{proof}
		($implies$) Suppose $A \times B \subseteq C \times D$. Let $a$ be an arbitrary element of $A$. Since $B$ is non-empty by lemma 3.1.5 there is an object $b$ contained in $B$. We can construct the ordered pair $(a,b)$ which will be contained in 
		$A \times B$ and hence by assumption also in $C \times D$ but then $a$ must be contained in $C$ and therefore $A \subseteq C$. A similar argument shows that $B \subseteq D$. ($\impliedby$) Suppose that $A \subseteq C$ and $B \subseteq
		D$. Let $z$ be an arbitrary element $A \times B$ then $z$ is an ordered pair $(x,y)$ with $x \in A$ and $y \in B$ but by assumption we have then $x \in C$ and $y \in D$ thus $z \in C \times D$ and the result follows as desired.
	\end{proof}
	\lined
	
	\begin{remark}
		The second part of the question in the text asks what happens when the hypotheses that $A,B,C,D$ are all non-empty is dropped. In general then the assertion breaks. In particular the forward implication is not generally true. For example suppose that $A$ is the empty set then $A \times B = \emptyset \subseteq C \times D$ and $A \subseteq C$ for any set $C$. The breaking point is that $B$ need not be subset of $D$. Simply set $B \coloneqq \{1\}$ and $D \coloneqq \{2,3\}$ and we have a counterexample. Thus the requirement that all the sets be non-empty for the forward implication is necessary. Now let us consider the converse where we suppose that $A \subseteq C$ and $B \subseteq D$ and that at least one of the sets are empty.  We can't have $A$ and $B$ both non-empty since we assumed at least one of $A,B,C,D$ are empty thus either $C$ or $D$ is empty but if $C$ is empty then $A$ is empty and if $D$ is empty then $B$ is empty and we have a contradiction in either case thus $A$ or $B$ is empty. If this is the case then the result still holds since we have $A \times B = \emptyset \subseteq C \times D$. Thus for the converse statement we may drop the condition that all of the sets must be non-empty. 
	\end{remark}

	\newpage
	
	\begin{problem}[Exercise 3.5.7]
		Let $X,Y$ be sets, and let $\pi_{X \times Y \rightarrow X}:X \times Y \rightarrow X$ and $\pi_{X \times Y \rightarrow Y}:X \times Y \rightarrow Y$ be the maps $\pi_{X \times Y \rightarrow X}(x,y) \coloneqq x$ and $\pi_{X \times Y \rightarrow Y}(x,y) \coloneqq y$. These maps are know as the co-ordinate functions on $X \times Y$. Show that for any functions $f:Z \rightarrow X$ and $g:Z \times Y$, there exists a unique function $h:Z \rightarrow X \times Y$ such that  $\pi_{X \times Y \rightarrow X} \circ h = f$ and  $\pi_{X \times Y \rightarrow Y} \circ h = g$. The function $h$ is called the direct sum of $f$ and $g$ denoted $h = f \oplus g$
	\end{problem}
	
	\lined
	\begin{proof}
		We shall show that there is at most one function such that $\pi_{X \times Y \rightarrow X} \circ h = f$ and  $\pi_{X \times Y \rightarrow Y} \circ h = g$. Suppose that $h:Z \rightarrow X \times Y$ and $h':Z \rightarrow X \times Y$ are functions such that satisfying our requirement. Let $z$ be an arbitrary element of $Z$. Note that $h(z)$ is an ordered pair $(x,y)$ in $X \times Y$ and that $h'(z)$ is also an ordered pair $(x',y')$ in $X \times Y$. We have $(\pi_{X \times Y \rightarrow X} \circ h)(z) = (\pi_{X \times Y \rightarrow X} \circ h')(z)$ but that implies $x = x'$. Similarly using the fact that $\pi_{X \times Y \rightarrow Y} \circ h = 
		\pi_{X \times Y \rightarrow Y} \circ h'$ we have $y = y'$ and therefore we have $h(z) = h'(z)$ for all $z \in Z$. We have shown that there can be only at most one such function $h$ satisfying our conditions. We now proceed to show that there is at least such $h$. We define a function $h:Z \rightarrow X \times Y$ by $h(z) \coloneqq (f(z),g(z))$ for each $z \in Z$. Observe that
		
		\begin{equation*}
			\begin{split}
					(\pi_{X \times Y \rightarrow X} \circ h)(z) & = \pi_{X \times Y \rightarrow X}h(z) \\
					& = \pi_{X \times Y \rightarrow X}(f(z),g(z)) \\
					& = f(z)
			\end{split}
		\end{equation*}
	
		thus $\pi_{X \times Y \rightarrow X} \circ h = f$. A similar computation shows that $\pi_{X \times Y \rightarrow Y} \circ h = g$. 
		The result follows as desired.
		
		
	\end{proof}
	\lined
	
	\newpage
	
	\begin{problem}[Exercise 3.5.8]
		Let $n$ be a positive natural number and $X_1 \cdots X_n$ be sets. Show that $\prod_{i=1}^{n}X_i$ is empty
		if and only if at least one of the $X_i$ are empty.
	\end{problem}

	\lined
	\begin{proof}
		Suppose that $\prod_{i=1}^{n}X_i$ then appealing to lemma 3.5.12 we see that some $X_i$ must be empty. Now
		suppose that $j$ is a natural number such that $1 \leq j \leq n$ and $X_j$ is empty. Suppose for contradiction that
		$\prod_{i=1}^{n}X_i$ is non-empty then there is some ordered $n$-tuple $(x_i)_{1 \leq i \leq n}$ contained in $\prod_{i=1}^{n}X_i$.
		We have that $x_i \in X_i$ for $1 \leq i \leq n$ and thus in particular we have $x_j \in X_j$ which contradicts the fact that 
		$X_j$ is empty. Therefore $\prod_{i=1}^{n}X_i$ is empty and the result follows as desired.
	\end{proof}
	\lined
	
	\newpage
	
	\begin{problem}[Exercise 3.5.9]
		Suppose that $I$ and $J$ are two sets and for all $\alpha \in I$ that we have a set $A_{\alpha}$ and for all $\beta \in J$ that we have a set $B_{\beta}$. Show that $(\bigcup_{\alpha \in I}A_{\alpha}) \bigcap (\bigcup_{\beta \in J}B_{\beta}) = 
		\bigcup_{(\alpha,\beta) \in I \times J}(A_{\alpha} \cap B_{\beta})$.
	\end{problem}
	
	\lined
	\begin{proof}
		Let $z$ be an arbitrary element of $(\bigcup_{\alpha \in I}A_{\alpha}) \bigcap (\bigcup_{\beta \in J}B_{\beta})$ then we have that 
		$z$ is an element of $A_{\alpha}$ for some $\alpha \in I$ and that $z$ is also contained in $B_{\beta}$ for some $\beta$ in $J$ thus $z \in A_{\alpha} \cap B_{\beta}$ and $(\alpha,\beta) \in I \times J$ whence $z \in \bigcup_{(\alpha,\beta) \in I \times J}(A_{\alpha} \cap B_{\beta})$. Therefore we have $(\bigcup_{\alpha \in I}A_{\alpha}) \bigcap (\bigcup_{\beta \in J}B_{\beta}) \subseteq	\bigcup_{(\alpha,\beta) \in I \times J}(A_{\alpha} \cap B_{\beta})$. Now suppose that $z$ is an arbitrary element of
		$\bigcup_{(\alpha,\beta) \in I \times J}(A_{\alpha} \cap B_{\beta})$ then $z \in S$ for some $(\alpha,\beta) \in I \times J$ such
		that $S = A_{\alpha} \cap B_{\beta}$. We now see that $z \in \bigcup_{\alpha \in I} A_{\alpha}$ and that $z \in \bigcup_{\beta \in J} B_{\beta}$ thus $\bigcup_{(\alpha,\beta) \in I \times J}(A_{\alpha} \cap B_{\beta}) \subseteq (\bigcup_{\alpha \in I}A_{\alpha}) \bigcap (\bigcup_{\beta \in J}B_{\beta})$ and the result follows as desired.
		
	\end{proof}
	\lined
	
	\newpage
	
	\begin{problem}[Exercise 3.5.10]
		If $f:X \times Y$ is a function the graph of $f$ is the subset of $X \times Y$  defined by $(x,y) \in X \times Y: y = f(x)$
	\end{problem}
	
	\newpage
	
	\begin{problem}[Exercise 3.5.10]
		Let $f:X \times Y$ be a function. The graph of $f$
		is the subset of $X \times Y$ defined by 
		$\{(x,y) \in X \times Y:y = f(x)\}$. Show that for two 
		two functions $f:X \times Y$ and $\tilde{f} \times Y
		$ are equal if and only if they have the same graph.
		Conversely show that if $G$ is a subset of $X \times Y$ such that for each $x \in X$ there is exactly one ordered pair in $G$ with first component $x$ then there is exactly one function from $X$ to $Y$ whose graph is $G$.
	\end{problem}
	
	\lined
		\begin{proof}
			The proof is just a matter of unravelling the
			definition of the graph of a function and is hence 
			trivial. For the second part for each $x \in X$ define $f(x)$ to be the unique element in $Y$ such that
			$(x,f(x)) \in G$. This defines a function from $X$ to $Y$ whose graph is $G$. From the first part we know that there can only be at most one such function and the result follows.
		\end{proof}
	\lined
	
	\newpage
	
	\begin{problem}[Exercise 3.5.11]
		Show that the Power Set Axiom can be deduced from Lemma 3.4.10 and the other axioms of Set Theory. That is Lemma 3.4.10 can be used as an alternative formulation of the Power Set Axiom. 
	\end{problem}
	
	\lined
		\begin{proof}
			We shall just provide a sketch solution. Let $X$ and $Y$ be sets. Consider the Cartesian Product $X \times Y$. Appealing to Lemma 3.4.10 we have the set of all subsets of $X \times Y$ denoted $2^{X \times Y}$. Apply the Axiom of Specification on $2^{X \times Y}$ to get the set of all subsets of $X \times Y$ which obey the vertical line test. Finally using the Axiom of Replacement and the last part of Exercise 3.5.10 we obtain a set containing all functions from $X$ to $Y$.
		\end{proof}
	\lined
	

\end{document}