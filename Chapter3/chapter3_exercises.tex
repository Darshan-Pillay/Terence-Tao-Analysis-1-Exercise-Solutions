\documentclass{article}
\usepackage{amsthm}
\usepackage{mathtools}
\usepackage{fullpage}

\newcommand{\lined}{\noindent\rule{\textwidth}{1pt}}
\newtheorem*{problem}{Problem}
\newtheorem*{remark}{Remark}





\title{\huge Analysis 1: Chapter 3 \\ \LARGE {Set Theory Exercises}}
\date{}

\begin{document}
	\maketitle
	
	\begin{problem}[Exercise 3.4.6]
			Let $X$ be a set. Show that the collection of all subsets of $X$ is a set.
	\end{problem}
	
	\lined
		\begin{proof}
		Appealing to the Axiom of Infinity and the Pair Set Axiom we have that $\{0,1\}$ is a set. 
		By the Power Set Axiom there exists a set containing all functions from $X$ to $\{0,1\}$ which
		we shall denote as $\{0,1\}^X$. Define the property $P(f,S)$ pertaining to each $f \in \{0,1\}^X$ and
		any object $S$ to be the statement $S$ is a set and $S = f^{-1}(\{1\})$.  There is at most one set $S$ 
		such that $P(f,S)$ is true for each $f \in \{0,1\}^X$. Thus by the Axiom of Replacement
		$Q \coloneqq \{S:$ $S = f^{-1}(\{1\})$ for some $f \in \{0,1\}^X\}$ exists and is a set. All that remains to show
		is that every element of $Q$ is a subset of $X$ and every subset of $X$ is contained in $Q$. Let $Z$ be an arbitrary
		element of $Q$, then there exists some function $f \in \{0,1\}^X$ such that $Z = f^{-1}(\{1\})$ which is indeed a subset of
		$X$. Conversely let $Z$ be a subset of $X$. We shall show that there exists some function 
		$f \in \{0,1\}^X$ such that $Z = f^{-1}(\{1\})$. For each $x \in X$ define $f$ to be the mapping such that 
		$f(x) \coloneqq 1$ if $x \in Z$ and $f(x) \coloneqq 0$ if $x \notin Z$. This is clearly a function from $X$ to $\{0,1\}$ which has
		$Z = f^{-1}(\{1\})$ thus $Z \in Q$ and the result follows.
		
	\end{proof}
	\lined
	
	\newpage
	
	\begin{problem}[Exercise 3.4.7]
		Let $X$ and $Y$ be sets. A function $f:X' \rightarrow Y'$ from a subset $X'$ of $X$ to a subset $Y'$ of $Y$ is
		said to be a partial function from $X$ to $Y$. If $X$ and $Y$ are sets show that the collection of all partial functions
		from $X$ to $Y$ is a set.
	\end{problem}

	\begin{remark}
		The idea for the proof we shall give is analogous to a sort of double looping construct procedure in a
		computer programming language. In particular let $2^X$ and $2^Y$ be the collections of all subsets of $X$ and
		$Y$ respectively which are sets by the previous exercise. Informally, or each element of $2^X$ we loop through each element of
		$2^Y$ creating a set containing all sets of function spaces where a function space is just the set of all functions from some
		particular element of $2^X$ to some particular element of $2^Y$. Then we use the Axiom of Union to unbox all these sets of function spaces to get the set of all function spaces. Unboxing these sets in the same manner we get the set of all partial functions from $X$ to $Y$	\end{remark}

	\lined
		\begin{proof}
			Since $X$ and $Y$ are sets by the previous exercise asserts that the collection of all subsets of $X$ is a set and also that
			the collection of all subsets of $Y$ is set. Let $2^X$ and $2^Y$ denote these sets respectively. For each $X' \in 2^X$ and any object S define the property $P(X',S)$ to be the statement $S$ is a set such that for all objects $z$, we have 
			
			\[z \in S \iff z = Y'^{X'}\ \text{for some}\  Y' \in 2^Y.\]
			
			\noindent There is at most one set $S$ such that $P(X',S)$ is true for each $X' \in 2^X$. Thus by the Axiom of Replacement there exists a set $Q \coloneqq \{S:P(X',S)\ \text{is true for some}\ X' \in 2^X\}$. We observe that $Q$ is a family of sets thus
			by the axiom of the union the set $\bigcup Q$ exists. This set like $Q$ is also a family of sets. Thus applying the Axiom of Union once again we have that the set $\bigcup(\bigcup Q)$ exists. We claim that every element of $\bigcup(\bigcup Q)$ is a partial function from $X$ to $Y$ and that every partial function from $X$ to $Y$ is contained in $\bigcup(\bigcup Q)$.
			Let $f'$ be an arbitrary element of $\bigcup(\bigcup Q)$ then $f' \in T$ for some $T \in \bigcup Q$. But then any such $T$ will be contained in $H$ for	some set $H \in Q$. However if $H \in Q$ then for some $X' \in 2^X$	 we have for every object $K$ that $K \in H$ if and only if $K = Y'^{X'}\ \text{for some}\  Y' \in 2^Y$. Thus since $T \in H$ we have $T = Y'^{X'}$ for some 
			 $Y' \in 2^Y$. So that if $f' \in T$ then $f'$ is a partial function from $X$ to $Y$. We therefore see that every element of $\bigcup(\bigcup Q)$ is a partial function from $X$ to $Y$. Conversely let $f':X' \rightarrow Y'$ be a partial function from $X$ to $Y$ from a subset $X'$ of $X$ to a subset $Y'$ of $Y$. We shall show that $f' \in \bigcup(\bigcup Q)$. First note that by the Power Set Axiom the set $Y'^{X'}$ exists. Now for each $Y'' \in 2^Y$ let $P'(Y'',S')$ be the statement
			 $S'$ is a set such that $S' = Y''^{X'}$. For each $Y'' \in 2^Y$ there is at most one such set $S'$ thus by the Axiom of Replacement the set $H \coloneqq \{S':S' = Y''^{X'}\ \text{for some}\ Y'' \in 2^Y \}$ exists. Now observe that  $Y'^{X'} \in H$ and $H \in Q$ which implies that $Y'^{X'} \in \bigcup Q$ but $f' \in Y'^{X'}$ thus $f' \in \bigcup(\bigcup Q)$ and the result follows.
			 
		\end{proof}
	\lined
	
	\newpage
	
	\begin{remark}
		Exercise 3.5.1 in the third edition of the text consists of three parts even though all are listed as
		a single exercise.
	\end{remark}

	\begin{problem}[Exercise 3.5.1 (a)]
		Let $x$ and $y$ be any two objects (possibly the same). We define $(x,y) \coloneqq  \{\{x\},\{x,y\}\}$. Which can be shown to be a set that exists using Axiom 3.4 and the Axiom of Union. Show that $(x,y)$ obeys the usual property we would expect of an ordered pair (i.e two ordered pairs are equal if and only if all their components are equal)
	\end{problem}

	\lined
	\begin{proof}
		We shall show that for any objects $a,a',b\ \text{and}\  b'$ that $(a,b) = (a',b') \iff (a = a'\ \text{and}\ b = b')$.
		$(\implies)$ Suppose that $(a,b) = (a',b')$ then by definition we have that $\{\{a\},\{a,b\}\} = \{\{a'\},\{a',b'\}\}$. We will first show that $a = a'$. Since $\{a\} \in \{\{a\},\{a,b\}\}$ we have $\{a\} = \{a'\}$ or $\{a\} = \{a',b'\}$. If $\{a\} = \{a'\}$ then we are done. On the other hand if $\{a\} \neq \{a'\}$ then we must have $\{a\} = \{a',b'\}$ which implies that $\{a\} = \{a'\}$. Therefore in either case $\{a\} = \{a'\}$. Now we show that $\{b\} = \{b'\}$. From our hypothesis we have that $\{a,b\} \in \{\{a'\},\{a',b'\}\}$ and thus that
		$\{a,b\} = \{a'\}$ or $\{a,b\} = \{a',b'\}$. If $\{a,b\} = \{a'\}$ then $a = b = a'$. However $\{a',b'\} \in \{\{a\},\{a,b\}\}$ by our hypothesis so that $\{a',b'\} = \{a\}$ or $\{a',b'\} = \{a,b\}$ and in either case we have that $b' = b$. If on the other hand we have
		$\{a,b\} \neq \{a'\}$ then we must have $\{a,b\} = \{a',b'\}$ which implies that $b' = a$ or $b = b'$. If $b = b'$ we are done so we  consider the case when $b' = a$. Since $\{a,b\} \in \{\{a'\},\{a',b'\}\}$ we have that 
		$\{a,b\} = \{a'\}$ or $\{a,b\} = \{a',b'\}$ and in either case we have $b = b'$ since $a = a' = b'$.
		\linebreak
		
		\noindent $(\impliedby)$ The converse is trivial
	\end{proof}
	\lined
	
	\newpage
	
	\begin{remark}
		We will appeal to the previous part of this exercise (i.e we will use the result and definition of ordered pair in part a). The proof
		we will provide is similar to that of Exercise 3.4.7. 
	\end{remark}

	\begin{problem}[Exercise 3.5.1 (b)]
		Let $X$ and $Y$ be two sets. Show that the Cartesian Product of $X$ and $Y$ is a set (i.e that a set containing
		all ordered pairs whose first component is from $X$ and second is from $Y$ exists)
	\end{problem}

	\lined
	\begin{proof}
		For each $x \in X$ and any object $S$ define $P(x,S)$ to be the statement $S$ is a set such that for all objects $z$ we have
		$z \in S \iff z = \{\{x\},\{x,y\}\}$ for some $y \in Y$. For each $x \in X$ there is at most one set $S$ such that $P(x,S)$ is true. Thus by the Axiom of Replacement there is a set $\{S:P(x,S)\ \text{is true for some}\ x \in X\}$. By the Axiom of 
		Union we have that $Q \coloneqq \bigcup (\{S:P(x,S)\ \text{is true for some}\ x \in X\})$ is a set. We claim that every element of $Q$ is an ordered pair $(x,y)$ where $x \in X$ and $y \in Y$ and that every such ordered pair is in $Q$. First suppose that $z \in Q$ then $z \in T$ for some $T \in \{S:P(x,S)\ \text{is true for some}\ x \in X\}$. But then for this $T$ there exists some $x \in X$ such that for every object $q$ we have $q \in T$ if and only if $q = \{\{x\},\{x,y\}\}$ for some $y \in Y$. Thus we have $z = \{\{x\},\{x,y\}\}$ for some $x \in X$ and $y \in Y$ but by definition of an ordered pair for this $x$ and $y$ we have $(x,y) = \{\{x\},\{x,y\}\} = z$ thus every element of $Q$ is an ordered pair $(x,y)$ where $x \in X$ and $y \in Y$. Now suppose that $(x',y')$ is an ordered pair with $x' \in X$ and $y' \in Y$. For each $y \in Y$ and any object $q$ define $P'(y,q)$ to be the statement $q = \{\{x'\},\{x',y\}\}$. For each $y \in Y$ there is at most one object $q$ such that $P'(y,q)$ is true. Thus be the Axiom of Replacement there is a set $T \coloneqq \{q:P'(y,q)\ \text{is true for some}\ y \in Y\}$. Observe then that $P(x',T)$ is true thus $T \in \{S:P(x,S)\ \text{is true for some}\ x \in X\}$. Moreover $(x',y') = \{\{x'\},\{x',y'\}\} \in T$ thus $(x',y') \in Q$. The result follows as desired.
		 
	\end{proof}
	\lined



	

\end{document}