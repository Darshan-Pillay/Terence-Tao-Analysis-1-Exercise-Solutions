\documentclass{article}
\usepackage{amsthm}
\usepackage{mathtools}
\usepackage{fullpage}
\usepackage{amsfonts}

\newcommand{\lined}{\noindent\rule{\textwidth}{1pt}}
\newtheorem*{problem}{Problem}
\newtheorem{lemma}{Lemma}
\newtheorem*{disproof}{\emph{Disproof}}
\newtheorem*{remark}{Remark}





\title{\huge Analysis 1: Chapter 5 \\ \LARGE {The Real Numbers}}
\date{}

\begin{document}
	\maketitle
	
	\begin{remark}
		A common problem solving technique we'll see again and again is to start with the conclusion and work backwards to a point that we can conclude from our hypothesis. The solution to the problem below was obtained in this manner (try it!).
	\end{remark}
	
	\begin{problem}
		The sequence $(a_n)_{n = 1}^{\infty}$ defined by $a_n \coloneqq \frac{1}{n}$ is a cauchy sequence.
	\end{problem}
	
	\lined
		\begin{proof}
			Given a positive $\epsilon > 0$ we have that there exists a natural number $N$ such that $\frac{1}{\epsilon} \leq N$ which implies that $\epsilon \geq \frac{1}{N}$ and for all $j,k \geq N$ we have that $\lvert \frac{1}{j} - \frac{1}{k}\rvert \leq \frac{1}{N}$ and the result follows as desired.
		\end{proof}
	\lined
	
	\newpage
	
	\begin{problem}[Exercise 5.1.1]
		Every cauchy sequence is bounded.
	\end{problem}
	
	\lined
		\begin{proof}
		 Let $(a_n)_{n = 1}^{\infty}$ be a cauchy sequence. We have then that 	$(a_n)_{n}^{\infty}$ is eventually $1$-steady that is there exists 
		 a $N \geq 1$ such that for all $j,k$ we have $\lvert a_j - a_k \rvert\leq 1$. We can then split $(a_n)_{n = 1}^{\infty}$ into two parts $(a'_i)_{i = 1}^{N - 1}$ and $(b_n')_{n' = N}^{\infty}$. Observe that $(a'_i)_{i = 1}^{N - 1}$ is finite so it is bounded that is there exists some $M \geq 0$ such that $M \geq \lvert a_i \rvert$
for all $1 \leq i \leq N - 1$.  We also have that $(b_n')_{n' = N}^{\infty}$ is bounded since for $j \geq N$ we have that $\lvert b_j - b_N + b_N \rvert \leq  \lvert b_j - b_N \rvert + \lvert b_N \rvert \leq 1 + \lvert b_N \rvert$. Take the max of $1 + \lvert b_N \rvert$ and $M$ and the result follows.
		\end{proof}
	\lined
	
	\newpage
	
	
	\begin{problem}[Exercise 5.2.1]
		Let $(a_n)_{n = 1}^{\infty}$ and  $(b_n)_{n = 1}^{\infty}$ be sequences of rational numbers. Suppose $(a_n)_{n = 1}^{\infty}$ and  $(b_n)_{n = 1}^{\infty}$ are equivalent. Show that $(a_n)_{n = 1}^{\infty}$ is a cauchy sequence if and only if $(b_n)_{n = 1}^{\infty}$ and  $(b_n)_{n = 1}^{\infty}$ is a cauchy sequence.
	\end{problem}
	
	\lined
		\begin{proof}
			Let $\epsilon$ be a positive rational then $\frac{\epsilon}{3}$ is also positive. Suppose that $(a_n)_{n = 1}^{\infty}$ is a cauchy sequence then $(a_n)_{n = 1}^{\infty}$ is eventually $\frac{\epsilon}{3}$-steady so there exists a positive $N_1 \geq 1$ such that for all $j,k \geq N_1$ we have $\lvert a_j - a_k \rvert \leq \frac{\epsilon}{3}$. Also since $(a_n)_{n = 1}^{\infty}$ and  $(b_n)_{n = 1}^{\infty}$ are equivalent we have that there exists a positive $N_2 \geq 1$ such that for all $n \geq N_2$ we have $\lvert a_n - b_n \rvert \leq \frac{\epsilon}{3}$. 
We define $N \coloneqq max(N_1, N_2)$ and appealing to the triangle inequality we have that for all $j,k \geq N$ that 

	\begin{align*}
		\lvert b_j - b_k \rvert  & = \lvert b_j - a_j + a_j  - b_k + a_k - a_k \rvert \\ & \leq \lvert b_j - a_j \rvert  + \lvert a_k - b_k \rvert + \lvert a_j - a_k \rvert 
		 \\ & \leq \frac{\epsilon}{3} + \frac{\epsilon}{3} + \frac{\epsilon}{3} = \epsilon
	\end{align*}
	
	\noindent which shows that $(b_n)_{n = 1}^{\infty}$ is cauchy as desired.
		\end{proof}
	\lined

	
\end{document}